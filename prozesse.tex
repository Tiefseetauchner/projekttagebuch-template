\section{Prozesse}
\label{prozesse}

Die Festlegung von Prozessen ist ein wichtiger Teil des Projektmangagements. Deswegen werden hier die folgenden Prozesse festgelegt:

\begin{itemize}
	\item Kommunikation
	\item Designentscheidungen
	\item Entwicklungsprozesse
	\item Entwicklung Tooling
	\item Anforderungsänderungen
	\item Dokumentation
	\item Qualitätsmanagement
\end{itemize}

\subsection{Kommunikation}

Es gibt folgende Kommunikationswege:
\begin{itemize}
	\item {\color{red}(TODO)}
	\item {\color{red}(TODO)}
\end{itemize}

Jegliche Projektbezogene Kommunikation sollte wenn möglich über diese Applikationen erfolgen. 

Für Besprechungen in Person ist \textbf{kein} Protokoll zu führen, aber wenn wichtige Entscheidungen getroffen wurden, sollten diese in dem jeweiligen {\color{red}(TODO)} geposted werden.

\subsection{Designentscheidungen}

Für Designentscheidungen ist primär Design/UI/UX verantwortlich. Jedoch können und sollten diese in einem Plänum zumindest nachbesprochen werden.

Für Design soll primär {\color{red}(TODO)} verwendet werden, um das Design allen beteiligten zur Verfügung zu stellen.

\subsection{Entwicklungsprozesse}

Die Entwicklung folgt agile Taktiken. Die Entwicklung verläuft wie folgt:

\begin{itemize}
	\item Eine User Story wird in mehrere Slices unterteilt, die sich zunächst im Backlog befinden.
	\item Das Projektteam oder ein Teil des Projektteams erledigt das Backlog grooming.
	\item Slices werden der jeweilig zuständigen Person zugewiesen. Diese Person ist dann für Konzeption verantwortlich, kann dabei aber Unterstützung in Anspruch nehmen.
	\item Nach der Konzeption wird der Slice umgesetzt. Es werden keine Unit Tests benötigt, aber Unit Tests sind nie schlecht.
	\item Nach der Umsetzung gibt es von einer anderen Person ein Review, in der das Ergebnis noch einmal geprüft wird.
	\item Nach dem Review wird der Slice von QA getestet.
	\item Nachdem der slice getestet wurde, ist er für die Demo bereit. Wenn genug Slices in Demo hängen, wird ein Meeting ausgemacht, in dem die Änderung allen fachlich vorgestellt wird.
	\item Dannach ist der Slice closed und man beginnt wieder von vorne.
\end{itemize}

\subsection{Entwicklung Tooling}

\begin{itemize}
	\item Für die Versionsverwaltung wird {\color{red}(TODO)} verwendet.
	\item Für die Entwicklung wird {\color{red}(TODO)} verwendet.
	\item Für die Kompilierung und das Packaging wird {\color{red}(TODO)} verwendet.
	\item Für CI wird {\color{red}(TODO)} verwendet.
\end{itemize}

\subsection{Anforderungsänderungen}

Sollte es zu Anforderungsänderungen kommen, bzw. sollten diese benötigt sein, muss das Projektmanagement einen Vorschlag an das Plänum stellen. Dieser wird besprochen und hinterlegt.

\subsection{Dokumentation}

Der Code soll mit JavaDocs dokumentiert sein.

\subsection{Qualitätsmanagement}

Das Qualitätsmanagement wird nach dem Review und vor der Demo durchgeführt. Dabei sollte gleich nach der Konzeption eine Liste an Testfällen erstellt werden, die nacheinander abgearbeitet werden. Wenn alle diese Testfälle durchgehen ist der Slice bereit für die Demo. Wenn es einen Fehler gab, wird dieser mit der/dem Entwickler*in besprochen und behoben.